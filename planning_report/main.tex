%----------------------------------------------------------------------------------------
%	PACKAGES AND OTHER DOCUMENT CONFIGURATIONS
%----------------------------------------------------------------------------------------
\documentclass[a4paper,12pt]{article}
\usepackage[english]{babel}
\usepackage[latin1]{inputenc}
\usepackage{amsmath}
\usepackage{xcolor}
\usepackage{amssymb}
\usepackage{amsfonts}
\usepackage{graphicx}
\usepackage{geometry}
\usepackage{tikz}
\usepackage{enumitem}
\definecolor{link}{rgb}{0.4, 0.2, 0.2}
\geometry{
	a4paper,
	total={170mm,257mm},
	left=20mm,
	top=20mm,
}
\usepackage{xspace}
\usepackage[acronym]{glossaries}
\usepackage[nohyperlinks]{hyperref}
\hypersetup{
	unicode=false,          % non-Latin characters in Acrobat’s bookmarks
	pdftoolbar=true,        % show Acrobat’s toolbar?
	pdfmenubar=true,        % show Acrobat’s menu?
	pdffitwindow=false,     % window fit to page when opened
	pdfstartview={FitH},    % fits the width of the page to the window
	pdftitle={Research Planning Report},    % title
	pdfauthor={Aaron Panaitescu},     % author
	pdfkeywords={neuroplasticity, Parkinson's, stimulation}, % list of keywords
	pdfnewwindow=true,      % links in new PDF window
	colorlinks=true,       % false: boxed links; true: colored links
	linkcolor=link,          % color of internal links (change box color with linkbordercolor)
	citecolor=link,        % color of links to bibliography
	filecolor=link,      % color of file links
	urlcolor=link           % color of external links
}


\newcommand{\ballarrow}{\tikz[baseline=-0.6ex]{\draw[->] (0,0) -- (0.5,0); \filldraw (0.5,0) circle (2pt);}}

\begin{document}
\begin{titlepage}
	\newcommand{\HRule}{\rule{\linewidth}{0.5mm}}
	\setlength{\topmargin}{0in}
	\center

	\begin{flushleft} \large
		\begin{minipage}{0.4\textwidth}
			\includegraphics[scale=0.14]{imperial.png}
		\end{minipage}
	\end{flushleft}
	\vspace{4cm}

	%----------------------------------------------------------------------------------------
	%	TITLE SECTION
	%----------------------------------------------------------------------------------------
	\textbf{\large Literature review and thesis proposal}\\[0.1cm]
	{\large MRes. Neurotechonlogy}\\[0.5cm]

	\HRule \\[0.4cm]
	{\Large \bfseries Investigating plasticity in \\ Cortico-Basal Ganglia-Thalamus models \\ to improve stimulation-based treatments }
	\HRule \\[1cm]

	%----------------------------------------------------------------------------------------
	%	AUTHOR SECTION
	%----------------------------------------------------------------------------------------

	{\large Aaron Panaitescu \\
	(\textit{CID:} 02054726) \\[0.4cm]
	\textbf{Supervisor:} \\
	Hayriye Cagnan}

	%----------------------------------------------------------------------------------------
	%	DATE SECTION
	%----------------------------------------------------------------------------------------

	\vfill
	{Word Count: ????}\\[0.4cm]
	{\large \today}\\[0.8cm]
\end{titlepage}
%----------------------------------------------------------------------------------------
%	REPORT CONTENT
%----------------------------------------------------------------------------------------
\tableofcontents
\newpage

\section{Introduction / Abstract?}
**Stylistically, I would prefere to do an abstract here and breakdown the concepts in the following
section**
\section{Background and literature review}
**start this off nicely with a diagram of the research gap**

\subsection{Parkinson's Disease}
Parkinson's Disease (PD) is characterized by the loss of dopaminergic neurons and the presence of
\emph{Lewy Bodies} (LB) in the \emph{Substantia Nigra pars Compacta} (SNc)
\cite{del2018advances}. This dopaminergic depletion in the SNc leads to striatal dopamine
deficiency, which in turn leads to excessive synchronization in the basal ganglia. This
hyper-synchrony is a hallmark of the Parkinsonian state \cite{hammond2007pathological,
	helmich2012cerebral}, and thus, can be a promising therapeutic target.


\subsection{Deep brain stimulation: theory and practice}
Tremor is primarly generated in the Cortical-Basal Ganglia-Thalamic (CBGT) network. Thus, deep
brain stimulation (DBS) emerged as an effective tool in the treatment of advanced PD patients
\cite{del2018advances}. DBS work through the implantation of electrodes in particular regions
of the brain. The idea is that this sort of stimulation can be used to decorelate regios of the
basal ganglia and to push the network to a non-pathological state. There is significant evidence
that performing DBS in the STN or the GPi, can have major quality of life improvements in
PD patients, reducing tremor and restoring motor control \cite{rodriguez2005bilateral}. While
effective, DBS has its limitations.

Firstly, it is an highly-invasive procedure, and so it has many risks, though advances in
electrode technology and surgical techniques have reduced some of these risks.

Secondly, the stimulation needs to be on at all times; if it is turned off, patient simptoms
return.

Additionaly, the mechanism through which DBS removes tremors is not fully understood yet, there
are many wuestions left to answer. For example, why does DBS in PD get effective results around
130 Hz, when tremor frequency is on average around 20 Hz.

Finally, stimulating regions of the brain continously and at such high frequencies leads to
numerous side effects **citation desperatly needed**.

Therfore several areas of research emerge in improving stimulation-based treatments of PD:
finding non-invasive alternatives \cite{saturnino2017target, schwab2020spike}, eleciting lasting,
plastic changes in CBGT circuit, closing the loop and coupling stimulation to the presence of
patient symptoms \cite{beudel2018adaptive}, and improving stimulation patterns to harness the
exploit the rythmic patterns activity in the basal ganglia \cite{cagnan2017stimulating, west2022stimulating}.

\subsection{Stimulating at the right time}
The Parkinsonian state is modulated by the interplay between four pathways in the CBGT
the \textbf{hyperdirect} excitatory cortico-subthalamic pathway,
the \textbf{direct} inhibitory striato-pallidal pathway,
the \textbf{indirect} inhibitory subthalamic-pallido pathway and
the inhibitory \textbf{pallido-subthalamic} pathway. In the Parkinsonian state the hyperdirect
pathway is downregulated, while pallido-subthalamic pathway is upregulated, leading to a shift in
beta band oscillatory activity from sub-band $\beta_1$ to sub-band $\beta_2$, which corellates to
hyper-synchrony in the basal ganglia \cite{west2022stimulating}.
** \textbf{add diagram} **

**important** \cite{cagnan2017stimulating} \cite{beudel2018adaptive} \cite{west2022stimulating}

\subsection{Plasticity to recover network states}
**mention** \cite{lebedev2017brain} \cite{cramer2011harnessing}

\subsection{Neuron-level vs. Mean-field models}
**briefly, in general and expand in the context of plasticity**

**cover** \cite{jansen1995electroencephalogram} (\cite{hodgkin1952measurement} does this really need to be cited?)

**important** \cite{terman2002activity} \cite{rubin2004high} \cite{rubin2012basal} \\
\cite{duchet2023mean} \cite{shupe2021integrate} \cite{schwab2020spike}

\subsection{**Other ways of improving stimulation-based treatments**}
**stimulation parameter optimizations, closing the loop (e.g. aDBS \cite{beudel2018adaptive})**
\textit{
	Maybe this can be folded into the stimulating at the right time part
	since they are pretty closely related. Also should mention the idea that
	these methods are not mutually exclusive, meaning that in principle they
	could be combined, adaptive stim closing the loop, with inducing plastic
	changes as target
}

\section{Project Plan}


\subsection{Aims}
\begin{enumerate}
	\item Model neuroplasticity in a Parkinsonian CGBT network
	\item Investigate the viabillity of harnessing plasticity to remove the
	      system from the pathological state and analyze the dynamics that follow

	      **here i care about things like for how long and how does the network change.
	      To what degree can we induce changes etc.**

	      **should look into viable timescales + noise analyze how long does it take to rebound to
	      pathological state**
	\item Try to link potential results to potential stimulation protocols?
\end{enumerate}

\subsection{Methodolgy} **How indeed?** HH/IF Pakrkinsoni model + plasticity rules, trying
different stimulation-based protocols (link with experimental data?)

**look into how to tune IF model parameters from experimental data**

\subsection{Timeline}
**Gant chart thingy**



%**********************************************%
\newpage
%----------------------------------------------------------------------------------------
%	REFERENCES
%----------------------------------------------------------------------------------------
\addcontentsline{toc}{section}{References}
\bibliography{refs}
\bibliographystyle{apalike}
\newpage

%----------------------------------------------------------------------------------------
%	APPENDIX  
%----------------------------------------------------------------------------------------
\appendix
\section*{Appendices}
\addcontentsline{toc}{section}{Appendices}
\renewcommand{\thesubsection}{\Alph{subsection}}
\subsection{First appendix}
\subsection{Second appendix}

\end{document}
