%----------------------------------------------------------------------------------------
%	PACKAGES AND OTHER DOCUMENT CONFIGURATIONS
%----------------------------------------------------------------------------------------
\documentclass[a4paper,12pt]{article}
\usepackage[english]{babel}
\usepackage[latin1]{inputenc}
\usepackage{amsmath}
\usepackage{xcolor}
\usepackage{amssymb}
\usepackage{amsfonts}
\usepackage{graphicx}
\usepackage{geometry}
\usepackage{tikz}
\usepackage{enumitem}
\definecolor{link}{rgb}{0.4, 0.2, 0.2}
\geometry{
	a4paper,
	total={170mm,257mm},
	left=20mm,
	top=20mm,
}
\usepackage{xspace}
\usepackage[acronym]{glossaries}
\usepackage[nohyperlinks]{hyperref}
\hypersetup{
	unicode=false,          % non-Latin characters in Acrobat’s bookmarks
	pdftoolbar=true,        % show Acrobat’s toolbar?
	pdfmenubar=true,        % show Acrobat’s menu?
	pdffitwindow=false,     % window fit to page when opened
	pdfstartview={FitH},    % fits the width of the page to the window
	pdftitle={Research Planning Report},    % title
	pdfauthor={Aaron Panaitescu},     % author
	pdfkeywords={neuroplasticity, Parkinson's, stimulation}, % list of keywords
	pdfnewwindow=true,      % links in new PDF window
	colorlinks=true,       % false: boxed links; true: colored links
	linkcolor=link,          % color of internal links (change box color with linkbordercolor)
	citecolor=link,        % color of links to bibliography
	filecolor=link,      % color of file links
	urlcolor=link           % color of external links
}

\begin{document}


\begin{titlepage}
	\newcommand{\HRule}{\rule{\linewidth}{0.5mm}}
	\setlength{\topmargin}{0in}
	\center

	\begin{flushleft} \large
		\begin{minipage}{0.4\textwidth}
			\includegraphics[scale=0.14]{imperial.png}
		\end{minipage}
	\end{flushleft}
	\vspace{4cm}

	%----------------------------------------------------------------------------------------
	%	TITLE SECTION
	%----------------------------------------------------------------------------------------
	\textbf{\large Literature review and thesis proposal}\\[0.1cm]
	{\large MRes. Neurotechonlogy}\\[0.5cm]

	\HRule \\[0.4cm]
	{\Large \bfseries Investigating plasticity in Cortico-Basal Ganglia-Thalamus models for improving stimulation-based treatments }
	\HRule \\[1cm]

	%----------------------------------------------------------------------------------------
	%	AUTHOR SECTION
	%----------------------------------------------------------------------------------------

	{\large Aaron Panaitescu \\
	(\textit{CID:} 02054726) \\[0.4cm]
	\textbf{Supervisor:} \\
	Hayriye Cagnan}

	%----------------------------------------------------------------------------------------
	%	DATE SECTION
	%----------------------------------------------------------------------------------------

	\vfill
	{Word Count: ????}\\[0.4cm]
	{\large \today}\\[0.8cm]
\end{titlepage}
%**********************************************%
\tableofcontents
\newpage

\section{Introduction / Abstract?}
**Stylistically, I would prefere to do an abstract here and breakdown the concepts in the following section**
\section{Background and literature review}
**start this off nicely with a diagram of the research gap**

\subsection{Parkinson's Disease}
\textit{Outline}

** 1. Loss of SNc dopaminergic neurons.**

** 2. indirect GPe $\rightarrow$ STN pathway $\uparrow$,
hyperdirect Cortex $\rightarrow$ STN pathway $\downarrow$.
(dimmer switch model \cite{helmich2012cerebral}, \cite{west2022stimulating})**

** \textbf{add diagram} **

** 3. Hypersynchrony in the Basal Ganglia.**

\subsection{DBS: theory and practice}
**DBS as the state of the art in treatment**

**Limitations of DBS (invasiveness, side effects, it needs to be on permenently,
why 130Hz? when tremors are $\sim$ 20Hz)**

**citations needed**

$\Rightarrow$ plenty of things to be improved

\subsection{Stimulating at the right time?}
**important** \cite{cagnan2017stimulating} \cite{beudel2018adaptive} \cite{west2022stimulating}

\subsection{Plasticity to recover network states}
**mention** \cite{lebedev2017brain} \cite{cramer2011harnessing}

\subsection{Neuron-level vs. Mean-field models}
**briefly, in general and expand in the context of plasticity**

**cover** \cite{jansen1995electroencephalogram} (\cite{hodgkin1952measurement} does this really need to be cited?)

**important** \cite{duchet2023mean} \cite{shupe2021integrate} \cite{schwab2020spike}

\subsection{**Other ways of improving stimulation-based treatments**}
**stimulation parameter optimizations, closing the loop (e.g. aDBS \cite{beudel2018adaptive})**
\textit{
	Maybe this can be folded into the stimulating at the right time part
	since they are pretty closely related
}

\section{Project Plan}


\subsection{Aims}
\begin{enumerate}
	\item Model neuroplasticity in a Parkinsonian CGBT network
	\item Investigate the viabillity of harnessing plasticity to remove the
	      system from the pathological state and analyze the dynamics that follow

	      **here i care about things like for how long and how does the network change.
	      To what degree can we induce changes etc.**

        **should look into viable timescales for simulation**
	\item Try to link potential results to potential stimulation protocols?
\end{enumerate}

\subsection{Methodolgy} **How indeed?** HH/IF Pakrkinsoni model + plasticity rules, trying
different stimulation-based protocols (link with experimental data?)

\subsection{Timeline}
**Gant chart thingy**


%**********************************************%
\newpage
\addcontentsline{toc}{section}{References}
\bibliography{refs}
\bibliographystyle{apalike}
\newpage

\appendix
\section*{Appendices}
\addcontentsline{toc}{section}{Appendices}
\renewcommand{\thesubsection}{\Alph{subsection}}
\subsection{First appendix}
\subsection{Second appendix}

\end{document}
