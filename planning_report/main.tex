%----------------------------------------------------------------------------------------
%	PACKAGES AND OTHER DOCUMENT CONFIGURATIONS
%----------------------------------------------------------------------------------------
\documentclass[a4paper,12pt]{article}
\usepackage[english]{babel}
\usepackage[latin1]{inputenc}
\usepackage{amsmath}
\usepackage{xcolor}
\usepackage{amssymb}
\usepackage{amsfonts}
\usepackage{graphicx}
\usepackage{geometry}
\usepackage{tikz}
\usepackage{enumitem}
\definecolor{link}{rgb}{0.4, 0.2, 0.2}
\geometry{
	a4paper,
	total={170mm,257mm},
	left=20mm,
	top=20mm,
}
\usepackage{xspace}
\usepackage[acronym]{glossaries}
\usepackage[nohyperlinks]{hyperref}
\hypersetup{
	unicode=false,          % non-Latin characters in Acrobat’s bookmarks
	pdftoolbar=true,        % show Acrobat’s toolbar?
	pdfmenubar=true,        % show Acrobat’s menu?
	pdffitwindow=false,     % window fit to page when opened
	pdfstartview={FitH},    % fits the width of the page to the window
	pdftitle={Research Planning Report},    % title
	pdfauthor={Aaron Panaitescu},     % author
	pdfkeywords={neuroplasticity, Parkinson's, stimulation}, % list of keywords
	pdfnewwindow=true,      % links in new PDF window
	colorlinks=true,       % false: boxed links; true: colored links
	linkcolor=link,          % color of internal links (change box color with linkbordercolor)
	citecolor=link,        % color of links to bibliography
	filecolor=link,      % color of file links
	urlcolor=link           % color of external links
}

\usepackage{xspace} 

\newcommand{\ballarrow}{\tikz[baseline=-0.6ex]{\draw[->] (0,0) -- (0.5,0); \filldraw (0.5,0) circle (2pt);}}
\newcommand{\Na}{$Na^+$\xspace}
\newcommand{\K}{$K^+$\xspace}
\newcommand{\Ca}{$Ca^{2+}$\xspace}
\hyphenpenalty=750


\begin{document}
\begin{titlepage}
	\newcommand{\HRule}{\rule{\linewidth}{0.5mm}}
	\setlength{\topmargin}{0in}
	\center

	\begin{flushleft} \large
		\begin{minipage}{0.4\textwidth}
			\includegraphics[scale=0.14]{imperial.png}
		\end{minipage}
	\end{flushleft}
	\vspace{4cm}

	%----------------------------------------------------------------------------------------
	%	TITLE SECTION
	%----------------------------------------------------------------------------------------
	\textbf{\large Literature review and thesis proposal}\\[0.1cm]
	{\large MRes. Neurotechonlogy}\\[0.5cm]

	\HRule \\[0.4cm]
	{\Large \bfseries Investigating plasticity in \\ Cortico-Basal Ganglia-Thalamus models \\ to improve stimulation-based treatments }
	\HRule \\[1cm]

	%----------------------------------------------------------------------------------------
	%	AUTHOR SECTION
	%----------------------------------------------------------------------------------------

	{\large Aaron Panaitescu \\
	(\textit{CID:} 02054726) \\[0.4cm]
	\textbf{Supervisor:} \\
	Dr. Hayriye Cagnan}

	%----------------------------------------------------------------------------------------
	%	DATE SECTION
	%----------------------------------------------------------------------------------------

	\vfill
	{Word Count: ????}\\[0.4cm]
	{\large \today}\\[0.8cm]
\end{titlepage}
%----------------------------------------------------------------------------------------
%	REPORT CONTENT
%----------------------------------------------------------------------------------------
\tableofcontents
\newpage

\section{Introduction}
Parkinson's disease (PD) is characterized by pathological hypersynchrony in Cortico-basal Ganglia-Thalamic
(CBGT) circuit, driven by dopaminergic degeneration in the Substantia Nigra pars Compacta (SNc). While deep
brain stimulation (DBS) alleviates symptoms by disrupting synchrony, its mechanisms remain not well understood,
and therapeutic effects are transient, necessitating continuous stimulation. This goal of this study is to propose
a computational framework to investigate multi-site phase-locked stimulation in a spiking CBGT network model with
spike-timing-dependent plasticity (STDP) rules, aiming to induce lasting desynchronization through activity-dependent
synaptic reorganization.

Methodologically, pathological and healthy network regimes will be established through data-driven tuning of spiking
neuron models.
Systematic simulations then evaluate multi-site phase-locked stimulation across three experimental axes:
(1) STDP enabled/disabled in corticostriatal populations,
(2) stimulation applied to different pairs of basal ganglia nuclei and cortical afferents, and
(3) closed-loop adaptation to thalamic oscillatory activity.
The network will then be analyzed in terms of inter-populational coupling and stability of potential non-pathological
state with after stimulation.

This work advances the field in two key directions:
(1) integrates phase-locked stimulation with plasticity-targeted protocols, and
(2) evaluates stimulation targets across the CBGT circuit focusing on regions with experimentally validated STDP.
Results will inform clinical adaptive stimulation protocols that exploit neuroplasticity for lasting symptom relief,
while the methodology provides a generalizable computational framework for closed-loop neuromodulation design.

\section{Background and literature review}
**start this off nicely with a diagram of the research gap**

\subsection{Parkinson's Disease}
Parkinson's Disease (PD) is characterized by the loss of dopaminergic neurons and the presence of
\emph{Lewy Bodies} (LB) in the \emph{Substantia Nigra pars Compacta} (SNc)
\cite{del2018advances}. This dopaminergic depletion in the SNc leads to striatal dopamine
deficiency, which in turn leads to excessive synchronization in the basal ganglia.
Hyper-synchrony is a hallmark of the PD \cite{hammond2007pathological, helmich2012cerebral},
and thus, can be a promising therapeutic target.

**add a CBGT diagram**

\subsection{Deep brain stimulation: theory and practice}
Tremor is primarly generated in the Cortical-Basal Ganglia-Thalamic (CBGT) network.
Thus, deep brain stimulation (DBS) in this region emerged as an effective tool in
the treatment of advanced PD patients \cite{del2018advances}.
DBS in PD works through delivering high-frequency stimulation in regions of the CBGT
through implanted electrodes.
A prominent theory for this method's success is that it enables stimulation driven
decorelatation of the different BG nuclei, pushing it to a non-pathological state.
There is significant evidence that performing DBS in the STN or the GPi, can have major quality
of life improvements in PD patients, reducing tremor and restoring motor control
\cite{rodriguez2005bilateral, rubin2004high}.
DBS does, however have significant limitations.

Firstly, like any highly-invasive procedure, the implatation procedure has many risks.
Advances in electrode technology and surgical techniques have reduced some of these risks.
*cite*

Secondly, patient symptoms are only ameloriated while the stimulation is on. This means that
the device needs to be on continously, which can induce numerous side effects in patients
**citeee** and also creates the need for a continous power supply.

Finally, the mechanism through which DBS reduces tremors is not fully understood yet, there
are many questions left to answer. For instance, why does DBS in PD get effective results around
130 Hz, when tremor frequency is on average around 20 Hz.

Therfore, several areas of research emerge in improving stimulation-based treatments of PD:
finding non-invasive alternatives \cite{saturnino2017target, schwab2020spike}, eleciting lasting,
plastic changes in CBGT circuit, closing the loop and coupling stimulation to the presence of
patient symptoms \cite{beudel2018adaptive}, and improving stimulation patterns to harness the
exploit the rythmic patterns of activity in the basal ganglia \cite{cagnan2017stimulating, west2022stimulating}.

\subsection{Stimulating at the right time}
The Parkinsonian state is modulated by the interplay between four pathways in the CBGT
the \textbf{hyperdirect} excitatory cortico-subthalamic pathway,
the \textbf{direct} inhibitory striato-pallidal pathway,
the \textbf{indirect} inhibitory subthalamic-pallido pathway and
the inhibitory \textbf{pallido-subthalamic} (PS) pathway. In the Parkinsonian state the hyperdirect
pathway is downregulated, while pallido-subthalamic pathway is upregulated, leading to a shift in
beta band oscillatory activity from sub-band $\beta_1$ to sub-band $\beta_2$, which corellates to
hyper-synchrony in the basal ganglia \cite{west2022stimulating}.

\cite{cagnan2017stimulating} demonstrated that phase-specific neuromodulation can disrupt pathological
synchrony in essential tremor by delivering stimulation pulses locked to thalamic oscillatory phases.
Strategies like this and others \cite{beudel2018adaptive} show the potential for closed-loop,
adaptive therapies to address hypersynchrony in the BG.

\subsection{Plasticity to recover network states}
Neuroplasticity can be defined as "the ability of the nervous system to change its activity in
response to intrinsic or extrinsic stimuli by reorganizing its structure, functions, or
connections" \cite{mateos2019impact}.
Plasticity occurs at different scales in the brain, ranging from molecular to neural networks.
The focus of this project is synaptic plasticity, which occurs between a pre-synaptic and a
pos-synaptic neuron. While, the mechanisms of plasticity have been throughly studied **cite**,
this understanding is just starting to find its way into clinical applications in
neurorehabilitation (e.g. stroke, trauma, spinal cord injury \cite{cramer2011harnessing}) and
neuroprostheses \cite{lebedev2017brain}.

The potential of plasticity-based treatments in PD depends on the existance of
spike-timing-dependent plasticity (STDP) in the synapses of the different BG nuclei, particularly
in the STN \cite{rubin2012basal}. Long and short-term plasticity has been found in
corticostriatal synapses \cite{kreitzer2008striatal, di2009short} as well as cell-specific STDP in
striatal interneurons \cite{fino2010spike}.
\cite{thieu2024role} modeled STDP and random inputs in an STN neurons and found combining the two can
increase coupling between neurons.

Coordinated reset stimulation (CRS) delivers brief, phase-targeted electrical stimuli to specific
neuronal subpopulations within interconnected neural circuits.
CRS has been computationally modeled in a GPe-STN network to sucessfuly disrupt pathological
synchronization in the BG \cite{hauptmann2009cumulative, hauptmann2010restoration}.

\subsection{Modeling neural networks}
Most neuronal dynamics models are defined by defined by rules (ODEs) of how voltage in an unit changes
over time in response to voltage in the network and external current (e.g. noise, stimulation
current).
The question then becomes what is an unit? There are two broad categories branching out of this
question: An unit can be an individual neuron or a cluster of neurons. This section introduces
some common types of models, justifying the direction chosen for this project.

\paragraph{Neuron models} (or spiking models) work by combining networks of units that individually
simulate neuronal action potentials (APs).
These networks can then be organized into different nuclei (e.g. STN) and coherced to respect
excitatory or inhibitory connections between nuclei.
Within this class of models, different schemes for modeling neuronal dynamics exist,
such as Integrate-And-Fire (IF) \cite{gerstner2014if} and Hodgkin-Huxley
\cite{hodgkin1952measurement, gerstner2014hh}.

\paragraph{Mean-field models} simulate the local field potentials (LFPs) of clusters of neurons
and connect these clusters with population-level connectivity rules. (e.g.
\cite{jansen1995electroencephalogram, west2022stimulating})

% Neuron-level
\subsubsection{Integrate-And-Fire}
IF models treat action potentials as events.
This reduction is justified by the fact that the shape of APs is always \textit{approximately}
the same, meaning they \textit{cannot} convey information. \cite{gerstner2014if} describe a leaky IF model
is represented by an electrical circuit with a resistor($R$) and a capacitor($C$) in parallel
($I(t) = I_R + I_C$), combined with a reset condition when the potential exceeds threshold
$\theta$.
This can be modeled by,
\begin{align}
	\tau_m \dot u                                           & = -[u(t) - u_{rest}] + RI(t),                                    \\
	\intertext{Together with the reset condition,}
	\lim_{\delta \rightarrow 0; \delta > 0} u(t^f + \delta) & = u_r,                                                           \\
	t^f                                                     & = \{t | u(t)                                        = \theta \}.
\end{align}
Where $u$ is the membrane voltage, $u_{rest}$ is the resting potential, $u_r$ is the reset
potential, $\tau_m = RC$ is the time constant, and $I$ is external current.

These equations describe the update rules in a single unit. Connectivity rules between units
can be added to form neural networks. These connections can then be strengthend or weakend
depending on spike timings between units, modeling STDP. \cite{shupe2021integrate} simulate
plasticity in cortical columns; \cite{kromer2023synaptic} model plasticity between basal ganglia
nuclei both using a IF model.

\subsubsection{Hodgkin-Huxley}
The Hodgkin-Huxley (HH) model \cite{hodgkin1952measurement} simulates membrane potential by
considering the dynamics of different ion conductances.
Each ion has a reversal potential at which the force of the electrical and chemical gradients
across the membrane even out.
These can be modeled in an electrical circuit as a battery and the total channels of a
particular ion can be modeled as a resistor. Each modelled ion has a parallel branch in the
corresponding electrical circuit. The following equations adapted from \cite{gerstner2014hh}
describe the HH model:
\begin{align}
	I(t) & = I_C(t) + \sum_k I_k(t),
	\intertext{Where the current of each ion $k$ can be described in terms of the open probability
		of a $k$ channel, $p_k$ and the maximum conductance when all $k$ channels are open, $g_k$:}
	I_k  & = g_k p_k(u - E_k).
\end{align}
In particular Hodgkin and Huxley modeled $Na^+$, $K^+$ and a leak $L$ currents:
\begin{align}
	C \dot u = g_{Na}m^3h(E_{Na} - u) + g_K n^4(E_K - u) + g_L (E_L - u),
\end{align}
Where $m$, $h$ and $n$ are open probabilities for subunints of $Na$ or $K$ channels governed by:
\begin{align}
	\dot x = - \frac 1 {\tau_m(u)}[x - x_0(u)], &  & x \in \{m, h, n\}.
\end{align}
As in the IF model these rules describe the dynamics of a single unit, which can subsequently
be connected to model a networks of neurons (e.g. \cite{terman2002activity}) and augumented to
include plasticity rules \cite{borges2016effects}.

\subsubsection{Rubin-Terman}
\cite{terman2002activity} augumented the HH model to create a subthalamopallido network of the
basal ganglia (RT model), where GPe and STN neurons are modeled according following equation,
\begin{align}
	C \dot u = - I_L - I_K - I_{Na} - I_T - I_{Ca} - I_{AHP} - I_{s \rightarrow d} -
	I_{s \rightarrow s} + I_{app}
\end{align}
Where leak ($L$), \K, \Na and high-threshold \Ca currents \cite{song2000characterization}, are
described in HH-like equations, and low threshold calcium current $I_T$ is different for GPe and
STN neurons. $I_{AHP}$ represents \Ca-activated, voltage-independent \textit{afterhyperpolarization}
(AHP) $K^+$ current. $I_{s \rightarrow d}$ represents the inter-populational influence and is
defined by
$I_{s \rightarrow d} = \sum_{j} I_{ij}^{ds}$ where $s, d \in \{\text{GPe}, \text{STN}\}$.
$I_{ij}^{ds}$ is the current from the presynaptic neuron $j$ in $s$ to the
postsynaptic neuron $i$ in $d$, and is defined by
$I_{ij}^{ds} = w_{ij}^{ds}s_{ij}^{ds}(t - \tau_{ds})(u_i - E_{ds})$ where $w_{ij}$ is the
synaptic strength, $s_{ij}^{ds}$ is a synaptic variable and $\tau_{ds}$ is the transimission
delay between $d$ and $s$.
\cite{madadi2022inhibitory} expand on this model by adding inhibitory spike-timing-dependent
plasticity to this network with $\Delta w = \eta(\exp(-|\Delta t| / \tau_w) - \alpha)$,
where $\eta$ is the \textit{learning rate}, $\tau_w$ is the time constant of the plasticity
rule exponential decay and $\alpha$ is the depression factor.

% Population
\subsubsection{Other modeling approaches}
Neuronal populations can also be modeled with networks of coupled phase-oscillators as in
\cite{tass2006long}. \cite{duchet2023mean} modeled STDP as mean-field phase-locked plasticity in a network of
Kuramoto oscillators \cite{kuramoto1984phase}.

\cite{jansen1995electroencephalogram} model cortical columns of neurons by considering the PSPs of
excitatory and inhibitory populations in a column to replicate EEG recorded alpha and beta activity.
\cite{west2022stimulating} used these building blocks in full CBGT model to study effects on
phase-dependent stimulation on coupling.

\section{Project Plan}
The primary objective of this project is to investigate multi-site phase-locked stimulation in a spiking CBGT network
model, with the goal of inducing plastic changes that transition pathological network dynamics of the BG toward
a healthy desynchronized state. This work will address two important gaps in the current literature:
\begin{enumerate}[nosep]
	\item Integrating phase-locked stimulation with plasticity-targeted protocols, closing the loop in coordinated
	      reset paradigms.
	\item Systematically evaluating stimulation targets across the wider CBGT network, with emphasis on regions with
	      experimentally validated synaptic plasticity (e.g., corticostriatal connections) while accounting for
	      uncertainties in plasticity mechanisms at other nodes like the STN.
\end{enumerate}
The selection of a spiking network with STDP rules over phase-oscillator models with phase-dependent plasticity is
motivated by the following methodological consideration:
Simulating phase-locked stimulation within systems governed by \textit{a priori} phase-triggered plasticity introduces
inherent circularity.
Such coupling risks conflating stimulation mechanisms with network adaptation rules, effectively predetermining
intervention outcomes without much dependence on the physiological constraints.
In contrast, spiking networks with STDP decouple the synchronization mechanism from the phase-locked stimulation
preserving causal distinction between stimulation protocols and plasticity-driven reorganization.

\subsection{Methodology}
To achieve these objectives, the following steps are outlined:
\begin{enumerate}[nosep]
	\item Implement STDP mechanisms within a spiking CBGT network model.
	\item Tune pathological and healthy dynamical modes through data-driven parameter tuning, initially with plasticity disabled.
	\item Conduct systematic simulations of multi-site phase-locked stimulation under four experimental conditions:
	      \begin{itemize}[nosep]
		      \item STDP enabled/disabled in specific network nodes
		      \item Multi-site phase-locked stimulation applied to distinct BG nuclei and cortical afferents
	      \end{itemize}
	      Quantify network responses by analysing inter-nuclei coupling, and transitions between dynamical states.
	\item Derive stimulation protocols optimized for plasticity-driven network desynchronisation.
\end{enumerate}

\subsection{Timeline}
**Gant chart thingy**



%**********************************************%
\newpage
%----------------------------------------------------------------------------------------
%	REFERENCES
%----------------------------------------------------------------------------------------
\addcontentsline{toc}{section}{References}
\bibliography{refs}
\bibliographystyle{apalike}
\newpage

%----------------------------------------------------------------------------------------
%	APPENDIX  
%----------------------------------------------------------------------------------------
\appendix
\section*{Appendices}
\addcontentsline{toc}{section}{Appendices}
\renewcommand{\thesubsection}{\Alph{subsection}}
\subsection{First appendix}
\subsection{Second appendix}

\end{document}
